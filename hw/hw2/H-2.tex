\documentclass{article}
\usepackage{../fasy-hw}
\usepackage{ wasysym }

%% UPDATE these variables:
\renewcommand{\hwnum}{2}
\title{Computational Topology, Homework \hwnum}
\author{\todo{your name here}}
\collab{\todo{list your collaborators here}}
\date{due: 17 February 2022}

\begin{document}

\maketitle

\input{../directions}

\nextprob{Function Space}
% \collab{if applicable, update collab list}
Let $(X,d)$ be a metric space.  That is, $X$ is a set and $d \colon X \times X
\to \R$ is a distance metric on $X$.  A \emph{metric ball} at $x \in X$ with
radius $r \geq 0$ is the (open) set: $B_r(x) := \{x' \in X ~|~ d(x,x') < r \}$.
We can use the set of all metric balls to generate a
topology on $X$.  When we do so, we call this a \emph{metric} topology on $X$.

One of my favorite types of topological spaces are where the points represent
functions.  For example, let~$(X,\mathcal{T}_X)$ be a topological space
and let $(Y,\mathcal{T}_Y)$ be
a metric space (corresponding to the distance function~$d_Y
\colon Y \times Y \to \R$).  Let $C(X,Y)$ denote the set of all continuous
functions from~$X$ to~$Y$.  We can topologize~$C(X,Y)$ using the $L_{\infty}$-metric; that is,
we define a distance metric $\ell_{\infty} \colon C(X,Y) \times C(X,Y) \to
\R$~by
$$\ell_{\infty}(f,g) := \sup_{x \in X} d_Y(f(x),g(x)).$$

\begin{enumerate}[(a)]
    \item Prove that this is a metric.

        \paragraph{Answer}
        \todo{answer here}

    \item Suppose the topology on $X$ is the indiscrete topology; that is,
        $\mathcal{T}_X=\{\emptyset,X\}$.  Describe the topological space whose
        set is $C(X,\R)$ and whose topology is generated by metric balls in
        $\ell_{\infty}$. (Note: when I use $\R$ without explicitly stating the
        topology, please assume that we are using the standard topology).

        \paragraph{Answer}
        \todo{answer here}

\end{enumerate}

\nextprob{}
% \collab{TODO - uncomment if your collaborators on this have changed

We can also define an $L_p$-norm in Euclidean space:
$d_p \colon \R^d \times \R^d \to \R$ is defined by
$$d_p (x,y) = \left( \sum_{i=1,2, \ldots d} |x_i-y_i|^p \right)^{1/p}.$$
In fact, you should be quite familiar with this metric,
as $d_2$ is the Euclidean distance.  You might also know~$d_1$ as the
Manhattan distance.  We define $d_{\infty}(x,y) := \lim_{p \to \infty} d_p(x,y)
= \max_{i=1,2, \ldots d} |x_i - y_i|$.

Let $(X,\mathcal{T}_1)$ and $(X,\mathcal{T}_2)$ be two topological spaces.
We say that two $\mathcal{T}_1$ and $\mathcal{T}_2$ are \emph{equivalent topologies} if for each open set $A \in \mathcal{T}_1$,
there exists $A',A'' \in T_2$, where neither $A'$ nor $A''$ are $\emptyset$ or $X$
itself and $A' \subseteq A \subseteq A''$.  (And, symmetrically for $B \in
T_2$).  If you have a basis for your topology, you just need to prove that this
property holds on the basis elements.

Prove the following:

\begin{theorem}[Euclidean Space]
    Let $d \in \N$.  Let $(\R^d,\mathcal{T}_1)$ be the metric topology induced from the
    metric~$d_1$, and let $(\R^d,\mathcal{T}_{\infty})$ be the metric topology induced from the
    metric $d_{\infty}$.
   Then, $\mathcal{T}_1$ and $\mathcal{T}_2$ are equivalent topologies.
\end{theorem}

\begin{proof}

    \todo{}

\end{proof}

\nextprob{Planar Graph Coloring}
% \collab{TODO - uncomment if your collaborators on this have changed

Answer HE-CT Part I, Question 7 (Planar Graph Coloring).

\paragraph{Answer}
\todo{answer here}

\nextprob{Two-Coloring}
% \collab{TODO - uncomment if your collaborators on this have changed
Answer HE-CT Part II, Question~$2$~(\emph{$2$-Coloring}).

\paragraph{Answer}
\todo{answer here}

\nextprob{SoCG}
% \collab{TODO - uncomment if your collaborators on this have changed

The Symposium for Computational Geometry (SoCG), pronounced ``sausage'', is the
main conference in computational geometry and computational topology.
Choose a paper published in SoCG (any year).  If
you were a reviewer of that paper, then you would expect to provide a brief (1
paragraph) summary of the paper, highlighting the main contributions of the
paper.  Write a maximum of one page explaining the main result of the paper that
you chose.  In future assignments, we will continue to work on how to write a
review, so keep the opinions limited in this assignment.

If you need some inspiration on papers, here are some recommendations:
\begin{itemize}
    \item  Adamaszek, Adams,  Gasparovic, Gommel, Purvine, Sazdanovic,
        Wang, Wang, and Ziegelmeier. Vietoris-Rips and Cech Complexes of Metric
        Gluings. SoCG 2018.
    \item Amezquita, Quigley, Ophelders, Munch, and Chitwood.
        Quantifying Barley Morphology using Euler Characteristic Curves. SoCG
        2020.
    \item Chambers and Wang. Measuring Similarity Between Curves on 2-Manifolds
        via Homotopy Area. SoCG 2013.
    \item Driemel, Phillips, and Psarros.
        On the VC Dimension of Metric Balls under Fr\'echet and Hausdorff
        Distances. SoCG 2019
    \item Edelsbrunner and Osang. The Multi-cover Persistence of Euclidean
        Balls. SoCG 2018.
    \item Sheehy. The Persistent Homology of Distance Functions under Random
        Projection. SoCG 2014.
\end{itemize}

\paragraph{Answer}
\todo{answer here}

\end{document}

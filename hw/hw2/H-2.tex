\documentclass{article}
\usepackage{../fasy-hw}
\usepackage{ wasysym }

%% UPDATE these variables:
\renewcommand{\hwnum}{2}
\title{Computational Topology, Homework \hwnum}
\author{Rishi Borkar}
\collab{Braeden Sopp}
\date{due: 17 February 2022}

\begin{document}

\maketitle

\input{../directions}

\nextprob{Function Space}
% \collab{if applicable, update collab list}
Let $(X,d)$ be a metric space.  That is, $X$ is a set and $d \colon X \times X
\to \R$ is a distance metric on $X$.  A \emph{metric ball} at $x \in X$ with
radius $r \geq 0$ is the (open) set: $B_r(x) := \{x' \in X ~|~ d(x,x') < r \}$.
We can use the set of all metric balls to generate a
topology on $X$.  When we do so, we call this a \emph{metric} topology on $X$.

One of my favorite types of topological spaces are where the points represent
functions.  For example, let~$(X,\mathcal{T}_X)$ be a topological space
and let $(Y,\mathcal{T}_Y)$ be
a metric space (corresponding to the distance function~$d_Y
\colon Y \times Y \to \R$).  Let $C(X,Y)$ denote the set of all continuous
functions from~$X$ to~$Y$.  We can topologize~$C(X,Y)$ using the $L_{\infty}$-metric; that is,
we define a distance metric $\ell_{\infty} \colon C(X,Y) \times C(X,Y) \to
\R$~by
$$\ell_{\infty}(f,g) := \sup_{x \in X} d_Y(f(x),g(x)).$$

\begin{enumerate}[(a)]
    \item Prove that this is a metric.

        \paragraph{Answer}
          A function $l_\infty$ : C(X,Y) x C(X,Y) $\to R$ is said to be metric on C(X,Y) if:
          
          \begin{itemize}
          	\item $l_\infty(f, g) \geq 0$ for all f, g $\in C(X,Y)$ : $l_\infty(f, g)$ is the supremum of the distance $d_Y(f(x), g(x))$ for all $x \in X$. And since the distance is never zero, the supremum which is the smallest upperbound of it has to be $\geq 0$          	
          	\item $l_\infty(f, g) = 0$ if $f =g$ : If both f and g are the same point, the distance between the two points is zero. 
          	\item $l_\infty(f, g) = l_\infty(g, f)$ for all f, g $\in C(X,Y)$ : The distance between two points in the topological space remains the same no matter which direction it is measured from i.e $f(x) \to g(x)$ or $g(x) \to f(x)$          	
          	\item $l_\infty(f, g) \leq l_\infty(f, h) + l_\infty(h, g)$ for all f, g, h $\in C(X,Y)$ : If a third point $h$ is introduced, considering the triangle rule, $d_Y(f(x), g(x))$ will always be greater than  $d_Y(f(x), h(x))$ +  $d_Y(h(x), g(x))$ since C(X,Y) are continuous functions from X to Y. 
          \end{itemize}
                

    \item Suppose the topology on $X$ is the indiscrete topology; that is,
        $\mathcal{T}_X=\{\emptyset,X\}$.  Describe the topological space whose
        set is $C(X,\R)$ and whose topology is generated by metric balls in
        $\ell_{\infty}$. (Note: when I use $\R$ without explicitly stating the
        topology, please assume that we are using the standard topology).

        \paragraph{Answer}
        Since $\mathcal{T}_X=\{\emptyset,X\}$ is an indiscrete(trivial) topology, then $C(X,\R)$ is continuous for any topology in $\R$ because $f^{-1}(\emptyset) = \emptyset$ and $f^{-1}(X) = \R$, both of which are always open in any topology on $\R$.

\end{enumerate}

\nextprob{}
% \collab{TODO - uncomment if your collaborators on this have changed

We can also define an $L_p$-norm in Euclidean space:
$d_p \colon \R^d \times \R^d \to \R$ is defined by
$$d_p (x,y) = \left( \sum_{i=1,2, \ldots d} |x_i-y_i|^p \right)^{1/p}.$$
In fact, you should be quite familiar with this metric,
as $d_2$ is the Euclidean distance.  You might also know~$d_1$ as the
Manhattan distance.  We define $d_{\infty}(x,y) := \lim_{p \to \infty} d_p(x,y)
= \max_{i=1,2, \ldots d} |x_i - y_i|$.

Let $(X,\mathcal{T}_1)$ and $(X,\mathcal{T}_2)$ be two topological spaces.
We say that two $\mathcal{T}_1$ and $\mathcal{T}_2$ are \emph{equivalent topologies} if for each open set $A \in \mathcal{T}_1$,
there exists $A',A'' \in T_2$, where neither $A'$ nor $A''$ are $\emptyset$ or $X$
itself and $A' \subseteq A \subseteq A''$.  (And, symmetrically for $B \in
T_2$).  If you have a basis for your topology, you just need to prove that this
property holds on the basis elements.

Prove the following:

\begin{theorem}[Euclidean Space]
    Let $d \in \N$.  Let $(\R^d,\mathcal{T}_1)$ be the metric topology induced from the
    metric~$d_1$, and let $(\R^d,\mathcal{T}_{\infty})$ be the metric topology induced from the
    metric $d_{\infty}$.
   Then, $\mathcal{T}_1$ and $\mathcal{T}_2$ are equivalent topologies.
\end{theorem}

\begin{proof}

  	
  
\end{proof}

\nextprob{Planar Graph Coloring}
% \collab{TODO - uncomment if your collaborators on this have changed

Planar graph coloring (two credits). Recall that every planar graph has
a vertex of degree at most five. We can use this fact to show that every
planar graph has a vertex 6-coloring, that is, a coloring of each vertex with
one of six colors such that any two adjacent vertices have different colors.
Indeed, after removing a vertex with fewer than six neighbors we use
induction to 6-color the remaining graph and when we put the vertex back
we choose a color that differs from the colors of its neighbors. Refine the
argument to prove that every planar graph has a vertex 5-coloring.

Answer HE-CT Part I, Question 7 (Planar Graph Coloring).

\paragraph{Answer}

\begin{itemize}
	\item Let  $G$ be a planar graph and $v$ be a vertex with degree $\leq 5$ 
	\item If deg(v) $\leq 4$, color the neighbours of $v$ with different colors and color $v$ with any thats remaining
	\item If deg(v) $ = 5$ 
	\begin{itemize}
		\item Let the neighbours of $v$ be $v_1,v_2,...v_5$ and the colors assigned to them be $c_1,c_2,...c_5$
		\item Choose 2 alternate neighbours say $v_1$ and $v_3$ and consider the subgraph of all the vertices colored by their colours $c_1$ and $c_3$
		\item If the subgraphs of $v_1$ and $v_3$  are disconnected, colors in either of the subgraphs can be switched with each other so that $v_1$ and $v_3$ are both the same color. $v$ can then be colored with the remaining color.
		\item If the $v_1 - v_3$ subgraphs are connected, apply the same steps on  $v_2$ and $v_4$
		\item If the $v_2 - v_4$ subgraphs are connected as well, it means that the two subgraphs $v_1 - v_3$ \& $v_2 - v_4$  cross each other, which contradicts the planarity of the graph.		 
	\end{itemize}
\end{itemize}

\nextprob{Two-Coloring}
% \collab{TODO - uncomment if your collaborators on this have changed

2-coloring (two credits). Let K be a triangulation of an orientable 2-
manifold without boundary. Construct L by decomposing each edge into
two and each triangle into six. To do this, we add a new vertex in the
interior of each edge. Similarly, we add a new vertex in the interior of each
triangle, connecting it to the six vertices in the boundary of the triangle.
The resulting structure is the same as the barycentric subdivision of K,
which we will define in Chapter III.
(i) Show that the vertices of L can be 3-colored such that no two neighboring
vertices receive the same color.
(ii) Prove that the triangles of L can be 2-colored such that no two triangles
sharing an edge receive the same color.

Answer HE-CT Part II, Question~$2$~(\emph{$2$-Coloring}).

\paragraph{Answer}

\begin{itemize}
	\item 
\end{itemize}

\nextprob{SoCG}
% \collab{TODO - uncomment if your collaborators on this have changed

The Symposium for Computational Geometry (SoCG), pronounced ``sausage'', is the
main conference in computational geometry and computational topology.
Choose a paper published in SoCG (any year).  If
you were a reviewer of that paper, then you would expect to provide a brief (1
paragraph) summary of the paper, highlighting the main contributions of the
paper.  Write a maximum of one page explaining the main result of the paper that
you chose.  In future assignments, we will continue to work on how to write a
review, so keep the opinions limited in this assignment.

If you need some inspiration on papers, here are some recommendations:
\begin{itemize}
    \item  Adamaszek, Adams,  Gasparovic, Gommel, Purvine, Sazdanovic,
        Wang, Wang, and Ziegelmeier. Vietoris-Rips and Cech Complexes of Metric
        Gluings. SoCG 2018.
    \item  Amezquita, Quigley, Ophelders, Munch, and Chitwood. Quantifying Barley Morphology using Euler Characteristic Curves. SoCG
        2020
    \item Chambers and Wang. Measuring Similarity Between Curves on 2-Manifolds
        via Homotopy Area. SoCG 2013.
    \item Driemel, Phillips, and Psarros.
        On the VC Dimension of Metric Balls under Fr\'echet and Hausdorff
        Distances. SoCG 2019
    \item Edelsbrunner and Osang. The Multi-cover Persistence of Euclidean
        Balls. SoCG 2018.
    \item Sheehy. The Persistent Homology of Distance Functions under Random
        Projection. SoCG 2014.
\end{itemize}

\paragraph{Answer} 
 Amezquita, Quigley, Ophelders, Munch, and Chitwood. Quantifying Barley Morphology using Euler Characteristic Curves. SoCG
        2020 \\
        
The paper makes use of Support Vector Machines to classify barley varieties using both traditional data, topological data as well as a combination of the both. The traditional data mainly consists of data based on 11 quantifiable traditional shape descriptors. The topological data 





\end{document}
